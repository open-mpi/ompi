% -*- latex -*-
%
% Copyright (c) 2004-2005 The Trustees of Indiana University.
%                         All rights reserved.
% Copyright (c) 2004-2005 The Trustees of the University of Tennessee.
%                         All rights reserved.
% Copyright (c) 2004-2005 High Performance Computing Center Stuttgart, 
%                         University of Stuttgart.  All rights reserved.
% Copyright (c) 2004-2005 The Regents of the University of California.
%                         All rights reserved.
% $COPYRIGHT$
% 
% Additional copyrights may follow
% 
% $HEADER$
%

\chapter{Don't Panic! (Who Should Read This Document?)}

{\Huge JMS needs overhauling -- no such thing as a previous OMPI user}

This document probably looks huge to new users.  But don't panic!  It
is divided up into multiple, relatively independent sections that can
be read and digested separately.  Although this manual covers a lot of
relevant material for all users, the following guidelines are
suggested for various types of users.  If you are:

\begin{itemize}
\item {\bf New to MPI}: First, read Chapter~\ref{sec:introduction} for
  an introduction to MPI and Open MPI.  A good reference on MPI
  programming is also strongly recommended; there are several books
  available as well as excellent on-line tutorials
  (e.g.,~\cite{gropp98:_mpi2,gropp94:_using_mpi,gropp99:_using_mpi_2,snir96:_mpi_the_compl_refer}).
  
  When you're comfortable with the concepts of MPI, move on to {\bf
    New to Open MPI}.

  \vspace{-3pt}

\item {\bf New to Open MPI}: If you're familiar with MPI but unfamiliar
  with Open MPI, first read Chapter~\ref{sec:getting-started} for a
  mini-tutorial on getting started with Open MPI.  You'll probably be
  familiar with many of the concepts described, and simply learn the
  Open MPI terminology and commands.  Glance over and use as a reference
  Chapter~\ref{sec:commands} for the rest of the Open MPI commands.
  Chapter~\ref{sec:troubleshooting} contains some quick tips on common
  problems with Open MPI.

  Assuming that you've already got MPI codes that you want to run
  under LAM/MPI, read Chapter~\ref{sec:mpi-functionality} to see
  exactly what MPI-2 features LAM/MPI supports.
  
  When you're comfortable with all this, move on to {\bf Previous Open
    MPI user}.
  
  \vspace{-3pt}
  
\item {\bf Previous Open MPI user}: As a previous Open MPI user,
  you're probably already fairly familiar with all the Open MPI
  commands -- their basic functionality hasn't changed much.  However,
  many of them have grown new options and capabilities, particularly
  in the area of run-time tunable parameters.  So be sure to read
  Chapters~\ref{sec:ssi} to learn about Open MPI's Modular Component
  Architecture (MCA), Chapters~\ref{sec:mca-orte}
  and~\ref{sec:mca-ompi} (run-time environment and MPI MCA modules),
  and finally Chapter~\ref{sec:misc} (miscellaneous Open MPI
  information, features, etc.).
  
  If you're curious to see a brief listing of new features in this
  release, see the release notes in Chapter~\ref{sec:release-notes}.
  This isn't really necessary, but when you're kicking the tires of
  this version, it's a good way to ensure that you are aware of all
  the new features.
  
  Finally, even for the seasoned MPI and Open MPI veteran, be sure to
  check out Chapter~\ref{sec:debug} for information about debugging
  MPI programs in parallel.
  
  \vspace{-3pt}

\item {\bf System administrator}: Unless you're also a parallel
  programmer, you're reading the wrong document.  You should be
  reading the Open MPI Installation
  Guide~\cite{open_mpi_install_guide} for detailed
  information on how to configure, compile, and install Open MPI.
\end{itemize}

