% -*- latex -*-
%
% Copyright (c) 2004-2005 The Trustees of Indiana University.
%                         All rights reserved.
% Copyright (c) 2004-2005 The Trustees of the University of Tennessee.
%                         All rights reserved.
% Copyright (c) 2004-2005 High Performance Computing Center Stuttgart, 
%                         University of Stuttgart.  All rights reserved.
% Copyright (c) 2004-2005 The Regents of the University of California.
%                         All rights reserved.
% $COPYRIGHT$
% 
% Additional copyrights may follow
% 
% $HEADER$
%

\chapter{Release Notes}
\label{sec:release-notes}
\index{release notes|(}

This chapter contains release notes as they pertain to the run-time
operation of Open MPI.  The Installation Guide contains additional
release notes on the configuration, compilation, and installation of
Open MPI.

%%%%%%%%%%%%%%%%%%%%%%%%%%%%%%%%%%%%%%%%%%%%%%%%%%%%%%%%%%%%%%%%%%%%%%%%%%%
%%%%%%%%%%%%%%%%%%%%%%%%%%%%%%%%%%%%%%%%%%%%%%%%%%%%%%%%%%%%%%%%%%%%%%%%%%%

\section{Alpha Release}

Greetings!

The Open MPI development team is pleased to announce an alpha-quality
release of the current state of our software to a limited set of
"friends" -- those who are knowledgeable about MPI and can provide
intelligent feedback to us before a public release.  Our intent with
this release is to get the Open MPI software onto other people's
machines and see what creative ways you can come up with to break it,
and to send us your comments and suggestions.

However, we want to stress the following points:

\begin{itemize}
\item This is an {\em alpha} quality release.  We're quite aware that
  there are several things still broken (but report them anyway!).
  
\item Most notably, if you run basic performance testing, you'll
  notice that, for example, the GM numbers are still a microsecond or
  two too high (we've been concentrating on functionality for the last
  month -- performance tuning is coming shortly).
  
\item Since the competition in the HPC community is rather fierce,
  please do not redistribute this software without our permission.
  Also, please do not publish any results (good or bad) because, as
  mentioned above, this is pre-release software and we still have
  performance tuning to do.
\end{itemize}

%%%%%%%%%%%%%%%%%%%%%%%%%%%%%%%%%%%%%%%%%%%%%%%%%%%%%%%%%%%%%%%%%%%%%%%%%%%

\subsection{Release Frequency / Updates}

You have probably downloaded this tarball from
\url{http://www.open-mpi.org/nightly/}.  This directory is updated at
least once a day around 2am US/Indiana time (assuming that there is
new code to release).  It may be updated more frequently if a critical
bug fix is reported and fixed.

The best way to report bugs, send comments, or ask questions is to
sign up on the \url{mailto:devel@open-mpi.org} mailing list:

\centerline{\url{http://www.open-mpi.org/mailman/listinfo.cgi/devel}}

%%%%%%%%%%%%%%%%%%%%%%%%%%%%%%%%%%%%%%%%%%%%%%%%%%%%%%%%%%%%%%%%%%%%%%%%%%%

\subsection{Known Issues}

This tarball is an alpha release of Open MPI.  It is not yet complete,
mainly in the following areas:

\begin{itemize}
\item Support for Infiniband (both verbs and OpenIB) is missing
\item Support for Quadrics is missing
\item Support for Myrinet needs performance tuning
\item Support for MX needs performance tuning
\item Support for TCP needs performance tuning
\item Support for shared memory is not fully debugged
  
  If this becomes a problem during your testing, run the following:

\lstset{style=ompi-cmdline}
\begin{lstlisting}
shell$ rm -f <prefix>/lib/openmpi/*sm*
\end{lstlisting}
% Stupid emacs mode: $

  where \cmdarg{<prefix>} is the directory where you installed Open
  MPI.
  
\item Striping MPI messages across multiple networks is supported (and
  happens automatically when multiple networks are available), but
  needs performance tuning

\item The only run-time systems supported are:
  \begin{itemize}
  \item \cmd{rsh} / \cmd{ssh}
  \item BProc of the flavor that is used at Los Alamos National Labs (in
    particular, it must be used with the BJS scheduler)
  \end{itemize}

\item TotalView and other parallel debugger support is missing.
  
\item Complete user and system administrator documentation is missing
  (this file comprises the majority of the current user documentation)

\item The only systems that have been tested on are:
  \begin{itemize}
  \item Linux, 32 bit, with \cmd{gcc}
  \end{itemize}

\item Other systems have been lightly (but not fully tested):
  \begin{itemize}
  \item Linux, 64 bit, with \cmd{gcc}
  \item OS X (10.3), 32 bit, with \cmd{gcc}
  \end{itemize}

\item Missing MPI functionality:
  \begin{itemize}
  \item The Fortran 90 MPI API is disabled (it is not complete).
  \item MPI-2 dynamic functionality is temporarily broken.
  \item MPI-2 one-sided functionality will not be included in the
    first few releases of Open MPI
  \end{itemize}
  
\item After running MPI applications, the directory
  \file{/tmp/openmpi-sessions-<username>@<hostname>*} will exist (but
  will likely be empty).  It is safe to remove.

\end{itemize}


\index{release notes|)}
